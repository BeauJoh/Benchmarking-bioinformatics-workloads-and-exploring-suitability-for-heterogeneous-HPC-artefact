\documentclass[../document.tex]{subfiles}
\begin{document}\label{sec:related_work}

A major motivation for this work is based off the findings of Marjanovi et al.~\cite{marjanovic2016hpc} when showing that problem size matters when benchmark HPC super computer systems.

The Scalable Heterogeneous Computing benchmark suite as introduced by Lopez et al.~\cite{lopez2015examining} is another popular benchmark suite.
Unlike the previously mentioned benchmark suites, it supports multiple node benchmarking with an MPI implemented for multi-node parallelism. 
However adding the proposed various problem size support to SHOC would involve much more development time, since each application must be implemented in both CUDA and OpenCL.
Additionally SHOC offers various levels of performance tests, but very few level 2 real application kernels.
Finally SHOC was not selected due since the real application kernels offered currently lack classification into the dwarf taxonomy.

Barnes et al.~\cite{barnes2016evaluating} collected a set of applications representative of their current workload to evaluate performance on NERSC Cori phase 2.

Martineau et al.~\cite{martineau2016performance} collected a suite of benchmarks and three mini-apps to evaluate Clang OpenMP 4.5 support for NVIDIA GPUs.


\end{document}
