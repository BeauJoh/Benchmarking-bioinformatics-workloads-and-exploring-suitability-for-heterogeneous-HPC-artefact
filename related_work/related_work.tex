\documentclass[../document.tex]{subfiles}
\begin{document}\label{sec:related_work}

The Scalable Heterogeneous Computing benchmark suite as introduced by Lopez et al.~\cite{lopez2015examining} is another popular benchmark suite.
Unlike the previously mentioned benchmark suites, it supports multiple node benchmarking with an MPI implemented for multi-node parallelism. 
However adding the proposed various problem size support to SHOC would involve much more development time, since each application must be implemented in both CUDA and OpenCL.
Additionally SHOC offers various levels of performance tests, but very few level 2 real application kernels.
Finally SHOC was not selected due since the real application kernels offered currently lack classification into the dwarf taxonomy.

Barnes et al.~\cite{barnes2016evaluating} collected a representative set of applications from the current NERSC workload to guide optimization for Knights Landing in the Cori supercomputer.
As it is not always feasible to perform such a detailed performance study of the capabilities of different computational devices, the benchmarks described in this paper may give a rough understanding of device performance and limitations.

Martineau et al.~\cite{martineau2016performance} collected a suite of benchmarks and three mini-apps to evaluate Clang OpenMP 4.5 support for NVIDIA GPUs.

The NAS parallel benchmarks~\cite{bailey1991parallel} follow a `pencil-and-paper` approach, specifying the computational problem but leaving implementation choices such as language, data structures and algorithms to the user.
The benchmarks include varied kernels and applications which allow a more nuanced evaluation of a complete HPC system, however, the unconstrained approach does not readily support direct performance comparison between different hardware accelerators using a single set of codes.

\end{document}
