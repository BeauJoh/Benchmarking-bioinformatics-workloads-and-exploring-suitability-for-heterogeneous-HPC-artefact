\documentclass[../document.tex]{subfiles}
\begin{document}\label{ssec:software}

OpenCL version 1.2 was used for all experiments.
On the CPUs we used the Intel OpenCL driver version 6.3, provided in 16.1.1 and the 2016-R3 opencl-sdk release.
On the Nvidia GPUs we used the Nvidia OpenCL driver version 375.66, provided as part of CUDA 8.0.61, AMD GPUs used the OpenCL driver version provided in the amdappsdk v3.0.

The Knights Landing (KNL) architecture used the same OpenCL driver as the Intel CPU platforms, however, the 2018-R1 release of the Intel compiler was required to compile for the architecture natively on the host.
Additionally, due to Intel removing support for OpenCL on the KNL architecture, some additional compiler flags were required.
Unfortunately, as Intel has removed support for AVX2 vectorization (using the `{\tt -xMIC-AVX512}' flag), vector instructions use only 256-bit registers instead of the wider 512-bit registers available on KNL.
This means that floating-point performance on KNL is limited to half the theoretical peak.

GCC version 5.4.0 with glibc 2.23 was used for the Skylake i7 and GTX 1080,  
GCC version 4.8.5 with glibc 2.23 was used on the remaining platforms.
OS Ubuntu Linux 16.04.4 with kernel version 4.4.0 was used for the Skylake CPU and GTX 1080 GPU, Red Hat 4.8.5-11 with kernel version 3.10.0 was used on the other platforms.

As OpenDwarfs has no stable release version, it was extended from the last commit by the maintainer on 26 Feb 2016.~\cite{opendwarfs2017base}
LibSciBench version 0.2.2 was used for all performance measurements.

\end{document}
