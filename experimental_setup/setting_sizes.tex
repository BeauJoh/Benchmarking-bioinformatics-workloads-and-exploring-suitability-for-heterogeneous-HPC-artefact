\documentclass[../document.tex]{subfiles}
\begin{document}\label{ssec:setting_sizes}

%focus of this is the experimental evaluation to justify setting problem sizes
For each application 4 different problem sizes were selected, namely {\bf tiny}, {\bf small},{\bf medium} and {\bf large}.
This was based on the memory hierarchy of the Skylake CPU.
For instance, tiny should just fit within L1 cache, on the Skylake this corresponds to \SI{32}{\kibi\byte} of data cache, small should fit within the \SI{256}{\kibi\byte} L2 data cache, medium should fit within \SI{8192}{\kibi\byte} of the L3 cache, and large must be greater than \SI{8192}{\kibi\byte} to avoid caching and operate out of main memory.
This was verified using the {\tt lscpu} Linux tool.
\todo[inline]{Should the problem sizes be customized to memory hierarchy of each platform?}

Caching performance was measured using PAPI counters.
On the Skylake L1 and L2 data cache miss rates were counted using the {\tt PAPI\_L1\_DCM} and {\tt PAPI\_L2\_DCM} counters.
For L3 miss events, only the total cache counter event ({\tt PAPI\_L3\_TCM}) was available.
The final values presented as miss results are presented as a percentage, and were determined using the number of misses counted divided by the total instructions ({\tt PAPI\_TOT\_INS}).

The methodology to determine the appropriate size parameters is demonstrated on the k-means benchmark.
K-means is an iterative algorithm which groups a set of points into clusters, such that each point in is closer to the centroid of its assigned cluster than to the centroid of any other cluster.
Each step of the algorithm assigns each point to the cluster with the closest centroid, then relocates each cluster centroid to the mean of all points within the cluster.
Execution terminates when no clusters change size between iterations.
Starting positions for the centroids are determined randomly.
The OpenDwarfs benchmark previously required the object features to be read from a previously generated file.
We extended the benchmark to support generation of a random distribution of points.
This was done to increase true characteristics of cache performance, since repeated runs of clustering on the same feature space (loaded from file) would deterministically generate similar caching performance.
The same number of clusters is fixed, with the minimum and maximum cluster locations being returned both set to 5.

Given a fixed number of clusters, the parameters that may be used to select a problem size are the number of points $P_n$, and the dimensionality or number of features per point $F_n$.
In the kernel for k-means there are 3 large one dimensional arrays passed to the device, namely {\bf feature}, {\bf cluster} and {\bf membership}.
{\bf feature} is the array which stores the unclustered feature space, it is of size $P_n \times F_n \times \text{sizeof}\left(\text{float}\right)$.
Each feature is represented by a 32-bit floating point number.
{\bf cluster} is the working and output array to store the intermediately clustered points, it is of size $C_n \times F_n \times \text{sizeof}\left(\text{float}\right)$, where $C_n$ is the number of clusters.
{\bf membership} is an array indicating whether each point has changed to a new cluster in each iteration of the algorithm, it is of size $P_n \times \text{sizeof}\left(\text{int}\right)$, where $\text{sizeof}\left(\text{int}\right)$ is the number of bytes to represent an integer value.
Thereby the working kernel memory, in KiB, is:
\begin{equation}
    \frac{\text{size}\left(\textbf{feature}\right)+\text{size}\left(\textbf{membership}\right)+\text{size}\left(\textbf{cluster}\right)}{1024}
    \label{eq:kmeans_size}
\end{equation}

Since it is known for the target Skylake CPU that the L1 cache is of size \SI{32}{\kibi\byte}, and using the theoretical size of all memory buffers determined in Equation~\ref{eq:kmeans_size}, a careful experiment can be conducted around each benchmark and the appropriate amount of memory required.
The tiny problem size is defined to have 256 points and 30 features; from Equation~\ref{eq:kmeans_size} the total size of the main arrays is \SI{31.5}{\kibi\byte}, slightly smaller than the \SI{32}{\kibi\byte} L1 cache.
The number of points is increased for each larger problem size to ensure that the main arrays fit within the lower levels of the cache hierarchy, measuring the total execution time and respective caching events.
The first row of table~\ref{tab:problem_sizes} shows the number of points for each problem size for k-means.

Being constrained in such a way indicates that 30 points/objects should give the best L1, L2 and L3 in the {\bf tiny}, {\bf small} and {\bf medium} problem sizes respectively.
Spilling over at each level of cache should be at 31 points/objects for each problem size.
{\bf large} operates solely on main memory.
The large problem size is at least four times the size of the last-level cache - in the case of the Skylake, at least \SI{32}{\mebi\byte}. 

%To measure increase the number of points/objects by 1 from 5 to 47.\todo[inline]{How are number of points actually increased?}
%Configuration of the experiment is repeated 300 times, and each iteration of the kernel is instrumented for the respective cache events.

\add{figures for each level of cache and a brief write up.}

\begin{table*}[t]
\centering
\begin{threeparttable}
    \centering
    \caption{OpenDwarf workload scale parameters $\Phi$}
    \begin{tabular}{l|c|c|c|c}
        \bf Benchmark         & \bf tiny   & small  & medium     & large\\\hline
        {\tt kmeans}          & 256        & 2048   & 65536      & 131072\\
        {\tt lud}             & 80         & 240    & 1440       & 4096\\
        {\tt csr}             & 736        & 2416   & 14336      & 16384\\
        {\tt fft}$\medsquare$ & 128,16     & 128,128& 1024,512   & 2048,1024\\
        {\tt gem}             & 4TUT       & 2D3V   & nucleosome & 1KX5\\
        {\tt srad}            & 80,16      & 128,80 & 1024,336   & 2048,1024\\
        {\tt cfd}             & 128        & 1284   & 45056      & 193474
    \end{tabular}
    \begin{tablenotes}
    \item [$\medsquare$] {\tt fft} uses 2 arguments, shown here as {\tt arg1},{\tt arg2} these correspond to the square matrix size on which to perform a 2 dimensional Fast Fourier Transform. They are used within $\Phi$ using subscript notation, $\Phi_1$ and $\Phi_2$ respectively.
    \end{tablenotes}
    \label{tab:problem_sizes}
\end{threeparttable}
\end{table*}

For brevity, the procedures to used determine appropriate problem sizes each of the remaining benchmarks have been omitted, but the final selected parameters are presented in Table~\ref{tab:problem_sizes}.
The procedure to generate the remaining scale arguments is benchmark specific but was similar same as those presented for k-means, trial and error until a suitably sized domain for each cache size was found.

Gemnoui -- {\tt gem} is an n-body-method based benchmark which computes electrostatic potential of biomolecular structures.
Determining suitable problem sizes was performed by initially browsing the National Center for Biotechnology Information's Molecular Modeling Database (MMDB)~\cite{madej2013mmdb} and inspecting the corresponding Protein Data Bank format (pdb) files.
Molecules were then selected based on complexity, since the greater the complexity the greater the number of atoms required for the benchmark and thus the larger the memory footprint.
{\bf tiny} used the Prion Peptide 4TUT~\cite{yu2015crystal} and was the simplest structure, consisting of a single protein (1 molecule), it had the device side memory usage of \SI{31.3}{\kibi\byte} which should fit in the L1 cache (\SI{32}{\kibi\byte}) on the Skylake processor.
{\bf small} used a Leukocyte Receptor 2D3V~\cite{shiroishi2006crystal} also consisting of 1 protein molecule, with an associated memory footprint of 252KiB.
{\bf medium} used the nucleosome dataset originally provided in the OpenDwarf Benchmark Suite, using \SI{7498}{\kibi\byte} of device-side memory.
{\bf large} used an X-Ray Structure of a Nucleosome Core Particle~\cite{davey2002solvent}, consisting of 8 protein, 2 nucleotide, and 18 chemical molecules, and requiring \SI{10970.2}{\kibi\byte} of memory when executed by {\tt gem}.
Each {\tt pdb} file was converted to the {\tt pqr} atomic particle charge and radius format using the {\tt pdb2pqr}~\cite{dolinsky2004pdb2pqr} tool.
Generation of the solvent excluded molecular surface used the tool {\tt msms}~\cite{sanner1996reduced}.
The final datasets used for {\tt gem} and all other benchmarks can be found in this papers associated GitHub repository~\cite{johnston2017}.

Computational Fluid Dynamics -- {\tt cfd} is an unstructured grid benchmark which performs three-dimensional Euler equation computations to determine compressible flow (density, energy and momentum) over a surface.
This surface has a mesh data structure and internally is represented as a series of triangular vertices.
To generate suitable data sizes for this benchmark the data set provided with the benchmark {\tt fvcorr.domn.097K} has been shortened to $\Phi$ vertices, where $Phi$ is given in Table~\ref{tab:problem_sizes}, and was then stored as $\Phi${\tt.dat} files.
These files are then passed as the only argument into the benchmark.

\begin{table*}[t]
\centering
\begin{threeparttable}
    \centering
    \caption{Program Arguments}
    \begin{tabular}{l|l}
        \bf Benchmark & \bf Arguments\\\hline
        {\tt kmeans} & {\tt -g -f 26 -p} $\Phi$\\
        {\tt lud} & {\tt -s} $\Phi$\\
        $\Psi$ & {\tt createcsr -n} $\Phi$ {\tt -d 5000}$\medtriangleup$\\
        {\tt csr}\textdagger & {\tt -i} $\Psi$\\
        {\tt fft} & {\tt --2D  --pts1 } $\Phi_1$ {\tt --pts2} $\Phi_2$\\
        {\tt gem} & $\Phi$ {80 1 0}\\
        {\tt srad}& $\Phi_1$ $\Phi_2$ {\tt 0 127 0 127 0.5 1}\\
        {\tt cfd} & $\Phi${\tt.dat}\\
    \end{tabular}
    \begin{tablenotes}
    \item [$\medtriangleup$] The {\tt -d 5000} indicates density of the matrix in this instance 0.5\% dense (or 99.5\% sparse).
    \item [\textdagger] The {\tt csr} benchmark loads a file generated by each respective workload of $\Phi$, this file is generated by the {\tt createcsr} and is represented by the variable $\Psi$.
    \end{tablenotes}
    \label{tab:program_arguments}
\end{threeparttable}
\end{table*}

Where $\Phi$ is substituted as the argument for each benchmark, it is taken as the respective scale from Table~\ref{tab:problem_sizes} and is inserted into Table~\ref{tab:program_arguments}.

Each {\bf Device} can be selected in a uniform way between applications using the same notation, on this system {\bf Device} comprises of {\tt -p 1 -d 0 -t 0} for the Intel Skylake CPU, where {\tt p} and {\tt d} are the integer identifier of the platform and device to respectively use, and {\tt -p 1 -d 0 -t 1} for the Nvidia Geforce GTX 1080 GPU.
Each application is run as {\bf Benchmark} {\bf Device} {\tt --} {\bf Arguments}, where {\bf Arguments} is taken from Table~\ref{tab:program_arguments} at the selected scale of $\Phi$.
For reproducibility the entire set of python scripts with all problem sizes is available in the Github repository\cite{johnston2017}. 

\end{document}
