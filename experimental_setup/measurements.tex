\documentclass[../document.tex]{subfiles}
\begin{document}\label{ssec:measurements}

%LibSciBench and performance counters
LibSciBench is a performance measurement tool which allows high precision timing events to be collected for statistical analysis~\cite{hoefler2015scientific}.
We used LibSciBench to record timings in conjunction with hardware events, which it collects via PAPI~\cite{mucci1999papi} counters.

We modified the applications in the OpenDwarfs Benchmark suite to insert library calls to LibSciBench to record timings and PAPI events for the three main components of application time: kernel execution, host setup and memory transfer operations. 
To help understand the timings, the following hardware counters were also collected:
\begin{itemize}
	\item total instructions and IPC (Instructions Per Cycle);
	\item L1 and L2 data cache misses;
	\item total L3 cache events in the form of request rate (requests/instructions) miss rate (misses/instructions) and miss ratio (misses/requests)
	\item data TLB (Translation Look-aside Buffer) miss rate (misses/instructions); and
	\item branch instructions and branch mispredictions.
\end{itemize}
\todo[inline]{Describe how we record energy}
The branching behavior of the code was characterized by branch rate (branch instructions/total instructions), branch misprediction rate (branch mispredictions/total instructions), and
branch misprediction ratio (branch mispredictions/branch instructions).
\todo[inline]{Do we really use each of these ratios?}

A sample of 300 iterations was collected for each measurement.
Only the kernel execution times/energy are presented in Figure\todo{reference the big figure}%~\ref{}. 

\end{document}
