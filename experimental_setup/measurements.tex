\documentclass[../document.tex]{subfiles}
\begin{document}\label{ssec:time}

%LibSciBench and performance counters
LibSciBench by Hoefler and Belli \cite{hoefler2015scientific} is measurement
tool which allows high precision timing events to be collected for later
statistical investigation. In addition, LibSciBench supports the use of
PAPI (The Performace API) \cite{mucci1999papi} events, thus additional
hardware event measurements were can also be collected by setting enviroment
variables {\tt LSB\_PAPI1} and {\tt LSB\_PAPI2} over the command line.

All applications presented in the OpenDwarfs Benchmark suite have had precise
measurements made by manually including LibSciBench library calls. This allowed
investigation into measured times for kernel execution, along with host setup
and memory transfer operations exibited in each application, but also the
culprits for each timing result were examined using these PAPI events. Thus,
also measurements of IPC (Instructions Per Cycle), the L1 and L2 data cache
and total L3 cache events in the form of request rate (requests/instructions),
miss rate (misses/instructions) and miss ratio (misses/requests) were also
taken. Other events collected were the data TLB (Translation Look-aside Buffer)
miss rate (misses/instructions), the branch rate (branch instructions/total
instructions), the branch misprediction rate (branch mispredictions/total
instructions) and branch misprediction ratio (branch mispredictions/branch
instructions).

\end{document}
