\documentclass[../document.tex]{subfiles}
\begin{document}\label{ssec:measurements}

%LibSciBench and performance counters
LibSciBench is a performance measurement tool which allows high precision timing events to be collected for statistical analysis~\cite{hoefler2015scientific}.
It offers a high resolution timer in order to measure short running kernel codes, reported with one cycle resolution and roughly \SI{6}{\nano\second} of overhead.
We used LibSciBench to record timings in conjunction with hardware events, which it collects via PAPI~\cite{mucci1999papi} counters.

We modified the applications in the OpenDwarfs benchmark suite to insert library calls to LibSciBench to record timings and PAPI events for the three main components of application time: kernel execution, host setup and memory transfer operations.
To help understand the timings, the following hardware counters were also collected:
\begin{itemize}
	\item total instructions and IPC (Instructions Per Cycle);
	\item L1 and L2 data cache misses;
	\item total L3 cache events in the form of request rate (requests/instructions) miss rate (misses/instructions) and miss ratio (misses/requests)
	\item data TLB (Translation Look-aside Buffer) miss rate (misses/instructions); and
	\item branch instructions and branch mispredictions.
\end{itemize}

A sample of 50 iterations was collected for each measurement.
Only the kernel execution times/energy are presented in the final results.

Energy measurements were taken on Intel platforms using the Running Average Power Limit (RAPL) PAPI module.
Similarly, a PAPI module was used in conjunction with the Nvidia Management Library (NVML) to measure the energy usage on Nvidia GPUs.

\end{document}
