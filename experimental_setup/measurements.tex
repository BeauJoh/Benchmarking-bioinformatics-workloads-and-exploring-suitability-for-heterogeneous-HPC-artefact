\documentclass[../document.tex]{subfiles}
\begin{document}\label{ssec:measurements}

%LibSciBench and performance counters

We measured execution time and energy for individual OpenCL kernels within each benchmark.
Each benchmark was run 50 times for each problem size (see \S\ref{ssec:setting_sizes}) for both execution time and energy measurements.
Applications that completed in less than two seconds were rerun until the time had elapsed with the mean execution times of each kernel were recorded, this precaution was taken to avoid measuring system events and other causes of noise along with events that correspond to the first run of a kernel such as cold cache on start-up etc.

A sample size of 50 per group -- for each combination of benchmark and problem size -- was used to ensure that sufficient statistical power $\beta = 0.8$ would be available to detect a significant difference in means on the scale of half standard deviation of separation.
This sample size was computed using the t-test power calculation over a normal distribution.

To help understand the timings, the following hardware counters were also collected:
\begin{itemize}
	\item total instructions and IPC (Instructions Per Cycle);
	\item L1 and L2 data cache misses;
	\item total L3 cache events in the form of request rate (requests / instructions), miss rate (misses / instructions), and miss ratio (misses/requests);
	\item data TLB (Translation Look-aside Buffer) miss rate (misses / instructions); and
	\item branch instructions and branch mispredictions.
\end{itemize}
For each benchmark we also measured memory transfer times between host and device, however, only the kernel execution times and energies are presented here.

Energy measurements were taken on Intel platforms using the RAPL PAPI module, and on Nvidia GPUs using the NVML PAPI module.

\end{document}
