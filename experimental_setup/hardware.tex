\documentclass[../document.tex]{subfiles}
\begin{document}\label{ssec:hardware}
\begin{table*}[t]
\caption{Hardware}
\resizebox{0.85\textwidth}{!}{\begin{minipage}{\textwidth}
\centering
\begin{threeparttable}
    \centering
    \begin{tabular}{l|c|c|c|c|c|c|c|c}
        Name         & Vendor   & Type  & Series    & \multicolumn{1}{m{1cm}|}{\centering Core \\ Count}     & \multicolumn{1}{m{2.5cm}|}{\centering Clock Frequency (\SI{}{\mega\hertz})\\(min/max/turbo)}  &\multicolumn{1}{m{2.1cm}|}{\centering Cache (\SI{}{\kibi\byte})\\ (L1/L2/L3)} & \multicolumn{1}{m{.8cm}|}{\centering TDP \\ (\SI{}{\watt})} &  \multicolumn{1}{m{1cm}}{\centering Launch\\ Date} \\ \hline
        Xeon E5-2697 v2  & Intel    & CPU   &Ivy Bridge & 24* &1200/2700/3500 & 32/256/30720 & 130 & Q3 2013\\
        i7-6700K & Intel    & CPU   &Skylake & 8* & 800/4000/4300 & 32/256/8192& 91 & Q3 2015\\
        Titan X & Nvidia & GPU & Pascal & 3584\textdagger & 1417/1531/NA & 48/2048/NA & 250 & Q3 2016\\
        GTX1080 & Nvidia & GPU & Pascal & 2560\textdagger & 1607/1733/NA & 48/2048/NA & 180 & Q2 2016\\
        GTX1080Ti & Nvidia & GPU & Pascal & 3584\textdagger & 1480/1582/NA & 48/2048/NA & 250 & Q1 2017\\
        K20m & Nvidia & GPU & Kepler & 2496\textdagger & 706/?/NA & 64/1536/NA & 225 & Q4 2012\\
        K40m & Nvidia & GPU & Kepler & 2880\textdagger & 745/875/NA & 64/1536/NA & 235 & Q4 2013\\
        Xeon Phi 7210 & Intel & MIC & KNL & 256\textdaggerdbl & 1300/1500/NA & 32/1024/NA & 215 & Q2 2016\\
    \end{tabular}
    \begin{tablenotes}
    \item [*] HyperThreaded cores
    \item [\textdagger] CUDA cores
    \item [\textdaggerdbl] Each physical core has 4 hardware threads per core, thus 64 cores
    \end{tablenotes}
\end{threeparttable}
\end{minipage}}
\label{tab:hardware}
\end{table*}

The experiments were conducted on eight hardware platforms and are presented in Table~\ref{tab:hardware}.
Two of these systems are Intel CPU architectures, five are current Nvidia GPUs and one is a MIC (Intel Knights Landing Xeon Phi).
The L1 Cache Size should be read as having both an instruction size cache and a data cache size of equal values as those displayed. 
For Nvidia GPUs, the L2 cache size reported is the size per SM.
For the Intel CPUs, hyper-threading was enabled and the frequency governor was set to `performance'.

\end{document}
