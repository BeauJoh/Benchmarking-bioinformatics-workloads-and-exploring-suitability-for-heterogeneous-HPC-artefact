\documentclass[../document.tex]{subfiles}
\begin{document}\label{sec:introduction}

The next generation of scientific High Performance Computing (HPC) systems will use a wide range of accelerators.
A single node may be heterogeneous, containing multiple different computing devices; moreover, a HPC system may offer nodes of different types.
For example, the Cori system at Lawrence Berkeley National Laboratory comprises 2,388 Cray XC40 nodes with Intel Haswell CPUs, and 9,688 Intel Xeon Phi nodes~\cite{declerck2016cori}.

\improvement{Include energy efficiency observations, perhaps from icpp paper?}

Recently in the supercomputing setting, from June 2016 to June 2017 the average energy efficiency of the top 10 of the Green500 supercomputers rose by 2.3x, from 4.8 to 11.1 gigaflops per watt.~\cite{feldman_2017}
This was in large part due to an instalment of the latest Nvidia Tesla P100 GPUs.
However, in the search for energy efficiency the diversity of the accelerators used won't stop at GPUs indeed it is set to increase.
IBM have started shipping Power9 CPUs with NVLINK and CAPI~\cite{morgan_2016}, a high bandwidth interconnect between Nvidia GPUs and other accelerator devices respectively, the first instalment was in July to the Summit supercomputer at Oak Ridge National Laboratories.\cite{morgan_2017}
Additionally, next generation promising architectures include Fujitsu's Post-K~\cite{morgan_2016_postk}, and Cray's CS-400, which is currently being installed as the Isambard system~\cite{feldman_2017_isambard}, both configurations heavily utilise ARM cores alongside other conventional accelerators -- several Intel Xeon Phi and Nvidia P100 GPUs per node.

Given added rising importance of energy efficiency in this space, the selection of most appropriate accelerator is presented with another criteria for selection or a second dimension of complexity.

Namely, should we select an accelerator which provides a shorter execution time or one that offers a higher energy efficiency?
Additionally, we ask do the two coincide or are these two criteria at odds for for particular or all modern scientific HPC workloads?
Given a choice of computing devices, does a method exist to select the most appropriate device for a particular computation?
Some vendors are removing support for OpenCL for HPC architectures, and tend to rely on OpenMP implementations for HPC systems, as shown with the Intel's Xeon Phi.
However having a common back-end, in the form of OpenCL, allows a direct comparision of identical code betweens diverse architectures - including FPGAs.

Additional curation effort has occurred to ensure reliability and scalability on many different architectures each across 4 various problem sizes of increasing computation complexity.

In Section~\ref{sec:related_work} we present related work.
The experimental setup is presented in Section~\ref{sec:experimental_setup}.
Performance measurements followed by a comparison of these results relative to each accelerator are presented in Section~\ref{sec:results}.
Finally, conclusions and future work are presented in Section~\ref{sec:conclusions} and Section~\ref{sec:future_work} respectively.
\end{document}
