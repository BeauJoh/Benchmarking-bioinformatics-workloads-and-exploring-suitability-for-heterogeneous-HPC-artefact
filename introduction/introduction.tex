\documentclass[../document.tex]{subfiles}
\begin{document}\label{sec:introduction}

The next generation of scientific High Performance Computer (HPC) systems promises to see an increasingly heterogeneous range of devices available as accelerators.
Several such systems have been proposed. \change{Mention Isambad and others}
An interesting observation here is that each node is expected to comprise different types of accelerators.
This poses an interesting question, does a method exist to select the most appropriate device?

\improvement{Include energy efficiency observations, perhaps from icpp paper?}

Given added rising importance of energy efficiency in this space, the selection of most appropriate accelerator is presented with another criteria for selection or a second dimension of complexity.
Namely, should we select an accelerator which provides a shorter execution time or one that offers a higher energy efficiency?
Additionally, we ask do the two coincide or are these two criteria at odds for for particular or all modern scientific HPC workloads?

In Section~\ref{sec:related_work} we present related work.
The experimental setup is presented in Section~\ref{sec:experimental_setup}.
Performance measurements followed by a comparison of these results relative to each accelerator are presented in Section~\ref{sec:results}.
Finally, conclusions and future work are presented in Section~\ref{sec:conclusions} and Section~\ref{sec:future_work} respectively.
\end{document}
