\documentclass[../document.tex]{subfiles}
\begin{document}\label{sec:introduction}

The next generation of scientific High Performance Computing (HPC) systems will use a wide range of accelerators.
A single node may be heterogeneous, containing multiple different computing devices; moreover, a HPC system may offer nodes of different types.
For example, the Cori system at Lawrence Berkeley National Laboratory comprises 2,388 Cray XC40 nodes with Intel Haswell CPUs, and 9,688 Intel Xeon Phi nodes~\cite{declerck2016cori}.
Several such systems have been proposed. \change{Mention Isambard and others}
Given a choice of computing devices, does a method exist to select the most appropriate device for a particular computation?

\improvement{Include energy efficiency observations, perhaps from icpp paper?}

Given added rising importance of energy efficiency in this space, the selection of most appropriate accelerator is presented with another criteria for selection or a second dimension of complexity.
Namely, should we select an accelerator which provides a shorter execution time or one that offers a higher energy efficiency?
Additionally, we ask do the two coincide or are these two criteria at odds for for particular or all modern scientific HPC workloads?

In Section~\ref{sec:related_work} we present related work.
The experimental setup is presented in Section~\ref{sec:experimental_setup}.
Performance measurements followed by a comparison of these results relative to each accelerator are presented in Section~\ref{sec:results}.
Finally, conclusions and future work are presented in Section~\ref{sec:conclusions} and Section~\ref{sec:future_work} respectively.
\end{document}
