\documentclass[../document.tex]{subfiles}
\begin{document}

For reasons of both performance and energy efficiency, high performance computing (HPC) hardware is becoming increasingly heterogeneous.
The OpenCL framework supports portable programming across a wide range of computing devices and is gaining influence in programming next-generation accelerators.
To characterize the performance of these devices across a range of applications requires a diverse, portable and configurable benchmark suite, and OpenCL is an attractive programming model for this purpose.

%Traditional HPC performance measures such as LINPACK lack the richness to characterize the performance of a heterogeneous system across the range of applications for which such a system may be used.
%A wider and more diverse suite of benchmarks is required, to more faithfully  represent the computational characteristics of scientific workloads.

We present an extended and enhanced version of the OpenDwarfs OpenCL benchmark suite, with a strong focus placed on robustness of applications, curation of additional benchmarks with an increased emphasis on correctness of results and choice of problem size.
Preliminary results and analysis are reported for eight benchmark codes on a diverse set of architectures -- three Intel CPUs, five Nvidia GPUs, six AMD GPUs and a Xeon Phi.

%Scientific application workloads typically contain one or more computationally intensive kernels, each of which may be categorized as one of the 13 ``dwarfs'', which represent patterns of computation and communication.
%We present an extensive performance study of 40 OpenCL application benchmarks representing 12 of the 13 dwarfs, on three different systems: a high-end CPU compute node; a previous-generation CPU compute node; and an embedded system.

%For a given set of accelerators in a HPC system, all kernels corresponding to a particular dwarf tend to perform best on the same accelerator.
%This means that a categorization of kernels according to dwarfs is an effective means to determine efficient scheduling of work to accelerators in HPC systems.

%Next paper title: Finding hidden Dwarfs in Kernels

\end{document}
