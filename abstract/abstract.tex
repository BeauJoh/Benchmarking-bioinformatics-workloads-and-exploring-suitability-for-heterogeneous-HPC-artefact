\documentclass[../document.tex]{subfiles}
\begin{document}

High performance computing hardware is becoming increasingly heterogeneous due to considerations of compute time and energy efficiency.
The OpenCL framework supports portable programming across a wide range of accelerators.

Scientific application workloads typically contain one or more computationally intensive kernels, each of which may be categorized as one of the 13 ``dwarfs'', which represent patterns of computation and communication.
We present an extensive performance study of 40 OpenCL application benchmarks representing 12 of the 13 dwarfs, on three different systems: a high-end CPU compute node; a previous-generation CPU compute node; and an embedded system.

For a given set of accelerators in a HPC system, all kernels corresponding to a particular dwarf tend to perform best on the same accelerator.
\todo[inline]{Best in terms of both time and energy?}
This means that a categorization of kernels according to dwarfs is an effective means to determine efficient scheduling of work to accelerators in HPC systems.

%Next paper title: Finding hidden Dwarfs in Kernels

\end{document}
