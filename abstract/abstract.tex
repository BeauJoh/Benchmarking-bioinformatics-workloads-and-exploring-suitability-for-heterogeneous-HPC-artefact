\documentclass[../document.tex]{subfiles}
\begin{document}

Hardware in the High Performance Computing space is becoming increasingly
heterogenous, with the range of accelerators supported by the OpenCL runtime
steadily increasing. This is on account of two major motivators, energy
efficiency and a reduction in computing time. All of the applications examined
in this setting are scientific, with computationally intensive regions of code
having already been partitioned into kernels. Each kernel will best be
described as one of the 13 dwarfs of scientific computation. 
This article presents an investigation  performed across 40 applications,
encompassing 12 of the 13 dwarfs, on 3 different systems namely, a current
high-end workstation system an older desktop system and an embedded system.
The primary finding presented in this work is that  generally a specific type
of accelerator will always outperform the others when running all applications
in a dwarf. The logical conclusion of this finding is that knowing the dwarf
hidden in a kernel can be used as the basis for the efficient scheduling of
work to accelerators in HPC systems.

%Next paper title: Finding hidden Dwarfs in Kernels

\end{document}
