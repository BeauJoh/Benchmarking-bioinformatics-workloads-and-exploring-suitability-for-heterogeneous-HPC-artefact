\documentclass[sigconf]{acmart}

\copyrightyear{2018} 
\acmYear{2018} 
\setcopyright{acmcopyright}
\acmConference[ICPP '18 Comp]{47th International Conference on Parallel Processing Companion}{August 13--16, 2018}{Eugene, OR, USA}
\acmBooktitle{ICPP '18 Comp: 47th International Conference on Parallel Processing Companion, August 13--16, 2018, Eugene, OR, USA}
\acmPrice{15.00}
\acmDOI{10.1145/3229710.3229729}
\acmISBN{978-1-4503-6523-9/18/08}

%remove acm stuff for arXiv.org
%\makeatletter
%\renewcommand\@formatdoi[1]{\ignorespaces}
%\makeatother
%\settopmatter{printacmref=false} % Removes citation information below abstract
%\renewcommand\footnotetextcopyrightpermission[1]{} % removes footnote with conference information in first column
%\pagestyle{plain} % removes running headers
%\acmDOI{}
\newcommand{\ignore}[1]{}
\usepackage{booktabs} % For formal tables
\usepackage{array}
\renewcommand{\arraystretch}{1.2}  % stretch space between table rows a bit
\usepackage[normalem]{ulem}
\usepackage{hyperref}
\usepackage{libertine}
\usepackage{subfiles}               %using subfiles
\usepackage{csvsimple}              %for turning csvs into tables
\usepackage{caption}                %comments in tables
\usepackage{subcaption}             %comments in subfigures
\usepackage{float}                  %in line table
\usepackage{amsfonts}               %for mathbb
\usepackage{amsmath}
\usepackage{graphicx}               %for including eps
\usepackage{epstopdf}               %^^
\usepackage{pbox}                   %for multiline cells in tables
\usepackage[boxed]{algorithm2e}     %for pseudocode
\usepackage{multirow}               %for tables
\usepackage{threeparttable}         %^^
\usepackage{xargs}                  % Use more than one optional parameter in a new commands
\usepackage{MnSymbol}               %funky citation symbols
\usepackage[binary-units=true]{siunitx}
\usepackage{capt-of}

%todonotes
\usepackage[colorinlistoftodos,prependcaption,textsize=small,disable]{todonotes}
\usepackage{marginnote}
\let\marginpar\marginnote
\newcommandx{\add}[2][1=]{\todo[inline,linecolor=blue,backgroundcolor=blue!25,bordercolor=blue,#1]{#2}}
\newcommandx{\unsure}[2][1=]{\todo[inline,linecolor=orange,backgroundcolor=orange!25,bordercolor=orange,#1]{#2}}
\newcommandx{\change}[2][1=]{\todo[inline,linecolor=red,backgroundcolor=red!25,bordercolor=red,#1]{#2}}
\newcommandx{\info}[2][1=]{\todo[inline,linecolor=olive,backgroundcolor=olive!25,bordercolor=olive,#1]{#2}}
\newcommandx{\improvement}[2][1=]{\todo[inline,linecolor=violet,backgroundcolor=violet!25,bordercolor=violet,#1]{#2}}
\newcommandx{\thiswillnotshow}[2][1=]{\todo[disable,#1]{#2}}
%

\hyphenation{Open-Dwarfs}

\def\sharedaffiliation{%
\end{tabular}
\begin{tabular}{c}}
%

\begin{document}
\title[Dwarfs on Accelerators]{Dwarfs on Accelerators: Enhancing OpenCL Benchmarking for Heterogeneous Computing Architectures}

%\author{Author details omitted for review}
\author{Beau Johnston}
\orcid{0000-0001-5426-1415}
\affiliation{%
    \department{Research School of Computer Science}
	\institution{Australian National University}
	\city{Canberra} 
	\country{Australia}
}
\email{beau.johnston@anu.edu.au}

\author{Josh Milthorpe}
\orcid{0000-0002-3588-9896}
\affiliation{%
    \department{Research School of Computer Science}
	\institution{Australian National University}
	\city{Canberra} 
	\country{Australia}
}
\email{josh.milthorpe@anu.edu.au}

\begin{abstract}
	\subfile{abstract/abstract}
\end{abstract}

%\begin{CCSXML}
%	<ccs2012>
%	<concept>
%	<concept_id>10010520.10010521.10010542.10010546</concept_id>
%	<concept_desc>Computer systems organization~Heterogeneous (hybrid) systems</concept_desc>
%	<concept_significance>500</concept_significance>
%	</concept>
%	<concept>
%	<concept_id>10002944.10011123.10011124</concept_id>
%	<concept_desc>General and reference~Metrics</concept_desc>
%	<concept_significance>500</concept_significance>
%	</concept>
%	</ccs2012>
%\end{CCSXML}
%\ccsdesc[500]{Computer systems organization~Heterogeneous (hybrid) systems}
%\ccsdesc[500]{General and reference~Metrics}

\maketitle

\section{Introduction}\subfile{introduction/introduction}
\section{Enhancing the OpenDwarfs Benchmark Suite}\subfile{introduction/extending_the_opendwarfs_benchmark_suite}

\section{Related Work}\subfile{related_work/related_work}


\section{Experimental Setup}\label{sec:experimental_setup}
\subsection{Hardware}\subfile{experimental_setup/hardware}
\subsection{Software}\subfile{experimental_setup/software}
\subsection{Measurements}\subfile{experimental_setup/measurements}
\subsection{Setting Sizes}\subfile{experimental_setup/setting_sizes}

\section{Results}\label{sec:results}

The primary purpose of including these time results is to demonstrate the benefits of the extensions made to the OpenDwarfs Benchmark suite.
The use of LibSciBench allowed high resolution timing measurements over multiple code regions.
To demonstrate the portability of the Extended OpenDwarfs benchmark suite, we present results from 11 varied benchmarks running on 15 different devices representing four distinct classes of accelerator.
For 12 of the benchmarks, we measured multiple problem sizes and observed distinctly different scaling patterns between devices.
This underscores the importance of allowing a choice of problem size in a benchmarking suite.

\subsection{Time}\subfile{results/time}
\subsection{Energy}\subfile{results/energy}


\section{Conclusions}\subfile{conclusions/conclusions.tex}


\section{Future Work}\subfile{future_work/future_work.tex}

\section*{Acknowledgements}
We thank our colleagues at The University of Bristol's High Performance Computing Research group for the use of ``The Zoo'' Research cluster for experimental evaluation.


%References
\bibliographystyle{ACM-Reference-Format}
\bibliography{bibliography/bibliography}

\newpage
\listoftodos[Notes]

\end{document}
