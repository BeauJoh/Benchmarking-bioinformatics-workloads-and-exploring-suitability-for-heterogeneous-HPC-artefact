\documentclass[../document.tex]{subfiles}
\begin{document}\label{sec:conclusions}

We have performed essential curation of the OpenDwarfs benchmark suite.
We improved coverage of spectral methods by adding a new Discrete Wavelet Transform benchmark, and replacing the previous inadequate {\tt fft} benchmark.
All benchmarks were enhanced to allow multiple problem sizes; in this paper we report results for four different problem sizes, selected according to the memory hierarchy of CPU systems as motivated by Marjanovi{\'c}'s findings~\cite{marjanovic2016hpc}.
It is believed that these can now be quickly adjusted for next generation accelerator systems were each applications working memory will affect performance on these systems, this methodology was outlined in Section~\ref{ssec:setting_sizes}.

We ran many of the original benchmarks presented in the original OpenDwarfs~\cite{krommydas2016opendwarfs} paper but on current hardware.
This was done for two reasons, firstly to investigate the original findings to the state-of-the-art systems and secondly to extend the usefulness of the benchmark suite.
Re-examining the original codes on range of modern hardware showed limitations, such as the fixed problem sizes along with many platform-specific optimizations (such as local work-group size).
In the best case, such optimizations resulted in sub-optimal performance for newer systems (many problem sizes favored the original GPUs on which they were originally run).
In the worst case, they resulted in failures when running on untested platforms or changed execution arguments.

Finally a major contribution of this work was to integrate LibSciBench into the benchmark suite, which adds a high precision timing library and support for statistical analysis and visualization.
This has allowed collection of PAPI, energy and high resolution (sub-microsecond) time measurements at all stages of each application, which has added value to the analysis of OpenCL program flow on each system, for example identifying overheads in kernel construction and buffer enqueuing.

\end{document}
