\documentclass[../document.tex]{subfiles}
\begin{document}\label{sec:future_work}

We plan to complete analysis of the remaining benchmarks in the suite for multiple problem sizes, and to add a benchmark for one of the 13 dwarfs that is not currently represented in the suite (MapReduce).
A comparison to peak theoretical performance achievable for each application on each accelerator could offer additional perspective regarding each devices performance, instead of just a direct comparison.

Certain configuration parameters for the benchmarks, e.g. local workgroup size, are amenable to auto-tuning.
We plan to integrate auto-tuning into the benchmarking framework to provide confidence that the optimal parameters are used for each combination of code and accelerator.

The original emphasis of this research was in validating an intuitive hypothesis, namely, since all devices within a type of accelerator share similar physical characteristic and since applications which represent a dwarf have similar computation to communication patterns, we expect a type of accelerator to always outperform all other accelerator types over all applications within a dwarf.
Until this point, we were always lacking in benchmark suite with enough stable applications could be run, and adequately represented a dwarf.
Now that a flexible benchmark suite is in place and results can be generated quickly and reliably on a range of accelerators, the next effort will be placed on verifying of disproving this hypothesis.

\end{document}
