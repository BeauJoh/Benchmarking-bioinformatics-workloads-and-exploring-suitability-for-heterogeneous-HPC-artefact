\documentclass[../document.tex]{subfiles}
\begin{document}\label{sec:future_work}

We plan to complete analysis of the remaining benchmarks in the suite for multiple problem sizes, and to add a benchmark for one of the 13 dwarfs that is not currently represented in the suite (MapReduce).
In addition to comparing performance between devices, we would also like to develop some notion of `ideal' performance for each combination of benchmark and device, which would guide efforts to improve performance portability.

Certain configuration parameters for the benchmarks, e.g. local workgroup size, are amenable to auto-tuning.
We plan to integrate auto-tuning into the benchmarking framework to provide confidence that the optimal parameters are used for each combination of code and accelerator.

The original goal of this research was to discover methods for choosing the best device for a particular computational task, for example to support scheduling decisions under time and/or energy constraints.
Until now, we found the available OpenCL benchmark suites were not rich enough to adequately characterize performance across the diverse range of applications and computational devices of interest.
Now that a flexible benchmark suite is in place and results can be generated quickly and reliably on a range of accelerators, we plan to use these benchmarks to evaluate scheduling approaches.

\end{document}
